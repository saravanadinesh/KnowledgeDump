\chapter{Capital And Income}
Ever since modern economy started, people have been debating on this question: ``What share of profit of a venture should go to the investors and what share of it should go to the labourers''. Capitalism and Communism are just byproducts of these debates. Anyone trying to solve economic inequalities will, at some point or ther other, have to meditate upon this simple question. This question is also at the heart of a related debate about government regulations: How much government should regulate; Whether we should have a small government of a big government. And of course, for entrepreneurs this question will have to be answered before their venture can start. 

We will not attempt to answer this question in this chapter (people do PhDs on this!). Instead, we will explain the simple terms ``Capital'' and ``Income'', understanding which will equip the reader to attempt to answer this question.

\emph{Capital} is a stock. It corresponds to the total wealth owned at a given point in time `t' . In other words, it corresponds to wealth appropriation over all the previous times (previous years, previous months, days, whatever the time unit is) combined. Thus the unit of capital can be  USD, INR, GBP, gold  etc. A house, a car, furniture, factory machinery, money in fixed deposit/equity/mutual funds etc., are all different forms of capital. 

\emph{Income} is a flow. It corresponds to money made over a stipulated period of time - so it is not a quantity accumulated over a period of time. The unit of income can be $\frac{USD}{year}$,  $\frac{INR}{month}$, $\frac{GBP}{hr}$\ etc.

For an individual, her yearly income may include her income from salary as well as returns from mutual funds, rent from a house, interest from savings acc. Her income, thus, has two distinguishable parts: income from labour (salary) and income from capital (mutual fund returns, rent, savings interest etc.)

For a company, its income would equal its revenues per year. Of this, whatever is paid to the workers is the income from labour (Labourers got that money because it was ascertained as part of the revenue that was generated because of their labour, and hence one they deserved). Whatever is remaining in the revenue is the income from capital (The machinery they own, investment they received, land they had purchased etc.)

\section{Equations Pertaining to Capital and Income}
\begin{itemize}
	\item $ National Capital = Private Capital + Public Capital $
	\item $ National Capital = Domestic Capital + Net Foreign Capital $
	\begin{itemize}
		\item Net Foreign Capital for India is Capital in other countries owned by India \emph{minus} Capital in India owned by other countries 
	\end{itemize}
	\item $ National Income = Domestic Output + net income from Abroad $
	\item $ National Income = Capital Income + Labour Income $
	\item $ \alpha = r*\beta $
	\begin{itemize}
		\item $ \alpha $ = Capital's share in income. Unit = $ \% $
		\item $ r $= Rate of return on capital. Unit = $ \frac{\%}{time} $
		\item $ \beta $ = Capital to income ratio. Unit = $ time $
	\end{itemize}
\end{itemize}		

The last equation deserves a deeper explanation. Imagine a company A that produces 1 million $ \frac{\$}{year} $ of goods (and sells them successfully). Assume that to do this, they needed capital (office buildings, machinery, coffee machines, transportation vehicles etc.) worth 5 million \$ and they needed labourers (CEO, VPs, directors, managers, engineers etc.), worth 600K $ \frac{\$}{year} $. Then,
\begin{itemize}
	\item Their income due to labour would be $ 600K \frac{\$}{year} $
	\item Their income due to capital would be total revenue minus payment to labourers. i.e.,  $  (1\ million - 600\ K)  \frac{\$}{year} = 400 K \frac{\$}{year} $
	\item Rate of return on capital is $ \frac{400 \frac{\$}{year}}{5\ million\ \$} = 0.08 \frac{1}{year}, or 8 \frac{\%}{year} $
	\item So, in our formula, we have,
	\begin{itemize} 
		\item $ r $= Rate of return on capital = $ 8 \frac{\%}{year} $ 
		\item $ \beta $ = Capital to income ratio = $ \frac{5 million\ \$}{1\ million \frac{\$}{year}} = 5\ years $
		\item $ \alpha $ = Capital's share in income = $ \frac{400 \frac{\$}{year}}{1\ million\ \$} = 40\% $
	\end{itemize}
\end{itemize}			

\section{Expenditure and Value Depreciation} 
An individual is bound to spend part of his income (which is the sum of both income from capital and labour), for daily expenses such as food, entertainment  etc. This is expenditure. It is important to note that items that one purchases which can be sold back, ex., TV, fridge etc., should not be included in expenditure. Such items are just different forms of capital. The part of the income that an individual saves or converts into other types of capital (house, car etc.) gets added to his existing capital.

For a company, expenditure includes money spent on employees (their salary, benefits, coffee, lunch etc.,), energy expenses, machinery maintenance costs etc. Note that, out of this money spent on employees, their salary is referred to as ``income due to labour'' in the previous section. The rest of the company's expenses have to come from the ``income due to capital'' part of the revenue. \marginnote{What remains of an individual/company's income after expenditure gets added to existing capital}After these expenditures, whatever is left of the income from capital gets added to the existing capital. This is how capital grows over time. 

Any capital asset can appreciate or depreciate in value. The value of a house or gold could appreciate over time, while the value of a car will most likely depreciate over time, unless we are talking about classic cars. Now appreciation of capital should not be confused with income from capital. For instance, the appreciation in the value of a property is capital itself appreciating. But the rent collected from the property is income from capital. 

  