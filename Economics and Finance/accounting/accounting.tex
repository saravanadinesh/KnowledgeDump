\chapter{Accounting}
Income statement, balance sheet and cash flow statement are three important financial statements in accounting. 

\section{Income Statement}
Income statement shows the items related revenue, operational expenses and profit. \autoref{tab:incomeStmt} shows a sample income statement. 

	\begin{table}[h] \centering
	\arrStr{1.3}
		\begin{tabular}{l r r}
		\toprule
		Figure (INR)			& 2008 			& 2009 \\
		\midrule
		Net Sales			& 1,500,000		& 2,000,000	\\
		Cost of Sales		& (350,000)		& (375,000)	\\
		Gross Income		& 1,150,000		& 1,625,000 \\
		SG\&A				& (235,000)		& (260,000)	\\
		Operating Income	& 915,000		& 1,365,000	\\
		Other Income (Expense)		& 40,000		& 60,000   \\
		Extraordinary Gain (Loss)		& \textendash\ 	& (15,000) \\
		Interest Expense				& (50,000)		& (50,000) \\
		Net Profit Before Taxes\\(Pretax Income) 	& 905,000	& 1,360,000\\
		Taxes				& (300,000)		& (475,000) \\
		\midrule
		Net Income			& 605,000		& 885,000 \\
		\bottomrule
		\end{tabular}
	\caption{A Sample Income Statement}
	\label{tab:incomeStmt}
\end{table} 

\begin{itemize}
	\item \textbf{Net Sales:} Revenue from sale of goods. It is also referred to colloquially as \''Revenue\'' or \''Revenue from sales\''
	\item \textbf{Cost of Sales:} Cost of production. This includes cost of labour, raw materials, energy cost incurred towards production of goods etc. Depreciation (Capital value depreciation) can be accounted as part of the Cost of Sales only when the asset in question is directly associated with production - Otherwise it will be included in OpEx (see below). Cost of Sales is also referred to as COGS - Cost of Goods Sold
	\item \textbf{Gross Income:} Net Sales - Cost of Sales. 
	\item \textbf{Operating Expenses:} 
	\begin{itemize}
		\item SG\&A Selling, General and Administrative expenses: Includes advertisement cost, salaries to employees not directly involved in production likes sales personnel, executives and R\&D expenses\footnote{Whenever there is a confusion as to whether an expense is part of COGS or SG\&A, one can check if that expense varies, more or less, linearly with number of goods produced - If yes, then it is part of COGS. For example, in a car manufacturing plant, the salaries of labourers who put the car together will have to be added to COGS as producing more cars implies employing more such labourers or keeping them in the employ for a longer period of time}. Sometimes R\&D expenses are shown separately as an item. This is so as to give investors an idea about how much the company is spending on order to have a bright future. 
		\item Depreciation and Amortisation: Depreciation of assets not related to production) and amortization
		\item Other operating expenses
	\end{itemize} 
	\item \textbf{Operating Income:} Gross Income - Operating Expenses. 
	\item \textbf{Other Income:} Interest/Dividends from investments
	\item \textbf{Extraordinary Gain (Loss):} Something that isn\'t recurring (otherwise it would have been accounted for in one of the above bullet points.
	\item \textbf{EBIT}: Earnings before interests and taxes = Operating Income + Other Income + Extraordinary gain (loss)
	\item \textbf{Interest Expense:} Interest to be paid on loans to banks, on bonds.
	\item \textbf{Pretax Income:} EBIT - Interest expense
\end{itemize}

Apart from these there are some other terms dervied from the above terms. These are:
\begin{itemize}
\item\textbf{Gross Margin:} $$ \frac{Net\: Sales - Cost\: of\: Sales}{Net\: Sales} * 100\%$$
\item\textbf{Operating Margin:} $$\frac{Operating\: Income}{Net\: Sales}*100\%$$
\end{itemize}

\section{Cash Flow Statement}
The cash flow statement is distinct from the income statement and balance sheet because it does not include the amount of future incoming and outgoing cash that has been recorded on credit. It shows whether the company will be able to pay its obligations on time and hence won’t risk bankruptcy. Cash flow statement involves three components:

\begin{itemize}
\item \textbf{Core Operations} component of cash flow reflects how much cash is generated from a company's products or services. Generally, changes made in cash, accounts receivable, depreciation, inventory and accounts payable are reflected in cash from operations.
\item \textbf{Investing} component records cash outflow due to conversion of cash into other forms of capital (ex., machines, building) and cash inflow due to divestment (ex., selling machines, building)
\item \textbf{Financing} component records cash outflow when dividends are paid to investors and cash inflow when capital is raised from investors.
\end{itemize}

\section{Balance Sheet}
Balance items has three major items namely Assets, Liabilities and (Shareholer\'s) Equity. The following equation shows the relationships among them: 
$$Equity = Assets - Liabilities$$

\begin{itemize}
\item \textbf{Assets:} Includes all forms of money realizable 
	\begin{itemize}
	\item \textbf{Current Assets} are those that can be quickly converted into cash when needed. The most important ones are:
		\begin{itemize}
		\item \textbf{Cash:} Needs no explanation. More cash implies more liquidity and hence protection against tough times. It also means possibility of an investment at short notice (market for a new product suddenly opens up). Dwindling cash is a sign of trouble. But if lots of cash is lying around balance sheet after balance sheet, it means the company is struggling to grow or the management is short-sighted and do not know how to invest the cash.
		\item \textbf{Accounts Receivable:} Money that we are expecting to receive. This could be because we had sold our product on credit and soon the customer is going to pay for it. 
		\item \textbf{Inventory:} Raw materials purchased but not utilised, finished good that is not yet sold or any intermediate good not converted to finished product yet. Inventory turnover (cost of goods sold divided by average inventory) measures how quickly the company is moving merchandise through the warehouse to customers.If inventory is growing faster than sales, it signals deteriorating fundamentals. On the other hand, if the inventory isn\'t keeping up with sales, the company would miss out on sales opportunities. 
		\end{itemize}
	\item \textbf{Long Term Assets} contains
		\begin{itemize}
		\item \textbf{Fixed Assets:} Machinery, buildings owned etc. Usually the value reported for fixed assets are not easily verifiable and companies tend to inflate these figures. So investors don\'t pay too much attention to these
		\item \textbf{Intangible Assets:} Intellectual Property, Copyrights, Brand value, Good Will etc. 
		\end{itemize}
	\end{itemize}
\item \textbf{Liabilities:} Whatever the company owes to others - that is money it needs to pay, but has not paid yet. 
	\begin{itemize}
	\item \textbf{Current Liabilities:} 
		\begin{itemize}
		\item EMI type payments on debts
		\item Rent, wages owned etc.
		\end{itemize}
	\item \textbf{Long Term Liabilities:}
		\begin{itemize}
		\item \textbf{Long-term Debt:} Interest/capital payments on debts, bonds
		\end{itemize}
	\end{itemize}		
\item \textbf{Equity:} Sometimes called ``net assets''. Contains two important parts namely,
	\begin{itemize}
	\item \textbf{Paid-in Capital:} Money that was originally put in when the company started and directly invested at some other point in time.
	\item \textbf{Retained Earnings:} Money retained from net income (after paying dividends, if any) within the company to be reinvested.
	\end{itemize}
\end{itemize}

There is something called an Off-Balance Sheet Debt that investors should be aware of. This is a way companies use to hide away some debts. 

\subsection{Quick Ratio}
$Quick Ratio = \frac{Current Assets - Inventory}{Current Liabilities}$. If this is >= 1, it means the company has enough liquidity to cover their short term obligations. 
