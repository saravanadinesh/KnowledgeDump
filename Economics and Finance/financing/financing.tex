\chapter{Financing}
\section{Terminology}
	\begin{itemize}
	\item \textbf{Market Capitalization:} Outstanding stock * share price (of a company)
	\item \textbf{Creditor} is someone to whom the company owes money in the form of loan/bond interest/capital payment. In other words, the assets contributed by Creditors appear in the liabilities portion of the balance sheet.
	\item \textbf{Investor} is someone who owns equity in the company.  In other words, the assets contributed by the investor shows up in the equity portion of the balance sheet.
	\end{itemize}
	
\section{Methods of Financing}
The sources of financing are, generically, capital self-generated by the firm (retained earnings) and capital from external funders, obtained by issuing new debt and equity
	\begin{itemize}
	\item \textbf{Debt} \textit{from the company\'s PoV} is attractive when loan interest rates are low. The good thing about debt is that, it is a simple expense that does not change the ownership structure of the company. It appears as a predictable expense in the balance sheet. Since it is considered an expense, income taxes are calculated after debt payments are made. The downside of debt is that it does affect the cash flow statement, and when the company is going through hard times, the need to pay off creditors on time could create headaches. 
	\item \textbf{Debt} \textit{from the creditor\'s PoV} is attractive as it is less riskier than equity. When a company becomes bankrupt and is liquidated, creditors are first paid followed by preferred stock owners and if there is anything still left, only then are common stock owners paid. 
	\item \textbf{Equity} \textit{from the company\'s PoV} is attractive because, it is not obliged to pay the equity owners, i.e., investors, money on time - Basically investors are the owners of the company and if the company is losing money, it is same as owners losing money. The downside is that, equity owners will now have a say in how the company should be run, where it should invest its retained earnings etc. If company sells 50\% of its stock to someone early on when the company was not doing well, that someone will continue to own half of the company indefinitely even if the company\'s earnings have drastically improved over the years.
	\item \textbf{Equity} \textit{from the investor\'s PoV} is attractive because it usually has more returns than debt - this comes, of course, from the inherent risks involved. 
	\end{itemize}	
