\chapter{History of Economics}
\section{Agriculture - From Jungle to Civilization}
Before civilizations emerged, man was living in the jungle like any other animal. He foraged for food along with his small group of friends. He had no notion of a home, although he had some concept of a territory, which not different from other social animals like lions or monkeys. He had no notion of private possessions either - any weapon his group used was the common property of the entire group and did not belong to any single person. Everyone in his group had the same set of primitive skills such as hunting, making simple tools, clothes from animal skins etc. No one specialized in any particular skill. All of this changed with the invention of agriculture.

When people discovered that they could grow crops wherever they wanted and no longer had to forage for food, they settled down in one place\marginnote{Agriculture meant not wasting time foraging/hunting for food; people could spend time developing other skills leading to scientific advancement}. The notion of home, a permanent shelter, naturally developed as a result. Food security led to the aggressive expansion of their populations. Agriculture meant that a small number of people could produce food for many and hence the rest could divert their thoughts towards other things, such as developing building skills, making new tools etc. After a few hundred years into this process, some were dedicating their entire time to building houses, making furnitures and so on and only a portion of the population was involved in producing food.

\section{Private Property and Trade}
It is thought that even with all these specializations going on, for some time, people still had not developed concepts of private property. Everything people produced belonged to the entire community and was shared with everyone. Building houses, producing food etc. were still community activities. But as populations expanded, at some point, community activities became harder to organize, and the concept of private property and trading developed - Rice became the private property of farmers who grew them and furnitures became the private property of the carpenters who made them. And they exchanged their goods with one another. At this point, communities transformed into societies - after all, the word ``community'' comes from the root word for ``common'' and make sense only when there was only common property. 

Later in the 19th century, when an idea to abolish all private property was mooted, it came to called ``Communism''. Thus, in Communism, everything was common property that would belong to the state and every activity in the country will be organized by the state and the fruits of individual labour will be shared with everyone in the country - exactly the way primitive communities were, only in a much larger scale!!

\section{The Seeds of Modern Economics}
As this barter trading evolved, the idea that some goods were more worth than others developed. A good's worth was calculated based on:

\begin{enumerate}
	\item How much it was wanted by others - people could have wanted food more than buildings. So the worth of food could have been more than the worth of buildings
	\item How much of time and energy one spent in producing it - perhaps making large quantities of food was relatively much easier than making buildings. So buildings could have in fact be held in higher worth than food
	\item  How much was the competition in producing the good - what if there were way too many builders and way too few farmers? Then, after all, food would have been of more worth than buildings!
\end{enumerate}

The first factor above constitutes what is called ``demand'' and the others two factors constitute what is called ``supply'' in modern economic terminology\marginnote{Supply \& Demand}. It is implicit that the worth of one good can only described relative to another good. For example, one unit of rice as equal to one piece of some standard furniture (say, a chair) or two units of milk. In other words, the ``exchange rate'' of rice to milk was 0.5 and rice to chair was 1 and so on. This way of trading one good for another is called barter trading and people traded this way for a while.

\subsection{Currency}
Barter trading had several problems. One can only imagine how cumbersome it would have been to remember these exchange rates of one's goods to so many other goods! Besides this, barter trading was inefficient - Suppose a farmer wanted a piece of furniture, but the carpenter would only accept milk as payment for his furniture (perhaps he already had enough rice). In this case, the farmer would have to first trade his rice with the milkman for some milk and then to trade this milk with the carpenter to get his furniture. To solve the problem of dealing with so many exchange rates and to increase efficiency in trading, the concept of currency was invented. The idea is that everyone valued their goods against a standard good, say gold. And if everyone then agrees to trade their goods in return for this standard good, the problems in barter trading vanishes - a farmer no longer needs to remember exchange rates of rice against all existing goods, but just a single exchange rate: that of rice to gold. Similarly, he doesn't have to first trade rice for milk and then trade milk for furniture - people would always pay him in gold and he can use gold to buy furniture.

One may ask why would a society choose gold or silver as the standard good, or as the currency. It is because an ideal currency would be something that is compact \& easy to carry, won't perish or degrade in some way, and be easy to store. Thus, gold is a good candidate for a currency because it was compact, it won't perish like rice and it won't catch rust.

Anyway, with a currency in place, a good's worth was no longer mentioned in the form of numerous exchange rates, but with just one ``price''. 
