\chapter{Macroeconomics}
Macroeconomics is about studying economics using country-wide/world-wide aggregate economic indicators such as national income, inflation, unemployment etc. 

\section{Macroeconomic Terms}
\subsection{Gross Domestic Product}
GDP is the monetary value of all the products and services coming from within the geographic borders of an entity (say, a country) over a stipulated period of time (say, an year). One can also state GDP in terms of national income: If we subtract income from foreign countries from the national income and then factor in capital value depreciation, we get a nation's GDP. 

GDP is usually specified in USD, although it can be specified in any other currencty - it is just that since USD usually holds in value, in an average sense, against all other currencies in the world, it is used as a universal unit for measuing GDP of a country. Having said that, even if the GDP of two countries are both listed in USD, it would be incorrect to compare, say Britain's GDP figure with India's GDP figure and infer how Britain is relatively wealthier than India. Such direct comparisons are meaningless because what 1 USD can buy in India is much more than what it can buy in Britain \footnote{It is separate topic in itself as to why this disparity exists. But just to give an idea, one may think of health care expenses as being cheaper in India than Britain because India has more doctors or doctors in India charge lesser fee or both.}. This is the topic of our next section.

\subsection{Purchasing Power}
From one country to another, there can be huge differences in what one can buy with a given amount of money. Let us consider an example of India vs. Britain. At today's (Apr 3, 2016) exchange rate, 1 GBP = 94 INR. But one can buy 4 times as much goods in India using 94 INR that what one could buy in Britain using 1 GBP. Similarly one can buy 3.333 times as much goods in India using 66 INR (1 USD = 66 INR), than what one could buy with 1 USD in the U.S. In other words, Indians have more ``purchasing power'' than Americans - if we assume that Indians always buy domestic goods and Americans do likewise.

It is important to remember that the process that is used to find out this purchasing power considers a select basket of goods. It is debatable whether a particular basket of goods make sense or if the idea that the same basket of goods mean the same thing in different countries is acceptable. For instance, in the U.S., whether there is a lack of public trasport, the prices of bus tickets for a given distance of travel could be several times than that in India. However, in the U.S., not many people value public transport as much as people in India do. So if the basket of goods contains the cost of public transportation, then obviously India will fare better compared to the U.S. - but such a comparison may not be useful at all.

In the previous section, we talked about GDP. That quantity is generally referred to as \emph{GDP (Nominal)}. \marginnote{GDP (Nominal) Vs. GDP (PPP)}In other words, it is indicative of a country's economy only in a nakesake sense. People usually factor the puchasing power into this GDP figure and come up with another quantity called \emph{GDP (PPP)}. Thus, India's GDP in 2014 was 2.051 trillion USD (nominal) and 7.411 trillion USD (PPP) \footnote{Since the U.S. is the standard economy against which the purchasing power of all countries are measured, the purchasing power conversion factor of USD is 1. Thus for the U.S., GDP (Nominal) = GDP (PPP)}.

\subsection{Wholesale Price Index and Consumer Price Index}
One of the indicators for the economic wellbeing of the people of a country is a weighted sum of the prices of a basket of goods in that country. If the prices considered are whole sale prices, then it is WPI and if they are prices at the retail shops, then it is CPI. Unlike GDP and PPP, WPI/CPI is not used to compare one country's economy against that of another. Hence different countries use different basket of goods and different weightages for them. Normally the chosen basket contains basic goods that every citizen would like to buy, thus including food, clothing, soap, fuel etc. 

\subsection{Inflation}
Inflation is closely related to PPP and WPI/CPI. It is a measure of by how much the WPI/CPI has increased (or decreased)  in one year as compared to the previous year or a chosen benchmark year \footnote{India used to use WPI to measure inflation. But we have recently switched to CPI.}. If inflation increases, it could mean that the  purchasing power of the common man has decreased. In general, when an economy sees huge development, it results in increased spending by citizens which would raise the demand on many goods leading to rise in inflation. If the fruits of economic development reached only a selective section of the people, then inflation could result in the sections that were left-out in the economic development to suffer - some of the items that they could afford ealier could become unaffordable due to rise in inflation. 

In countries where governments regulate the economy, the government usually takes measures to decrease inflation or keep it from increasing. \marginnote{Effects of govt. regulations on inflation} In India, RBI is tasked with controlling inflation. When inflation needs to be reduced, RBI increases interest rates - i.e., interest rates on the loans they give out to other banks. Consequently other banks also increase their loan interest rates. This ends up discouraging people to get loans and hence reduces their spending, thus bringing down the inflation. On the other hand, when RBI goes for ``rate cuts'', it encourages people to go out and spend. 

Measures to control inflation have both positive and negative effects. When RBI increases interest rates, the common man may benefit due to reduction in the prices of essential goods. But since it reduces spending, people will no longer give as much business to shops, theatres, real estate companies and that would result in decrease income to the said businesses. Consequently, in order to keep the profit from dropping, these companies could decide to cut down on expenses, which could then result in employees being laid off. Thus RBI has to walk a fine line to make the common man happy while preventing him from losing his job.

RBI rate increases/cuts have some side effect on people's investment behaviours. When RBI cuts interest rates, banks consequently follow. As a result, companies can go to banks for money via loans than go take money from people via bonds. So bonds become less attractive. But now that companies will have increased cash flow (since they can easily raise cash but getting loans from banks), they are expected to become more productive - so equities (shares) become more attractive. And when RBI increases rates, the exact opposite happens. 

\section{A Note of Caution on Macroeconomic Indicators}
As the name suggests, macroeconomic indicators give us only the big picture about the economy. When the GDP of a country increases, as compared to previous years or as compared to other countries, it indicates that the country, overall, is doing good. But it doesn't say anything about how equitable the economic growth has been. In other words, while the collective income of all the citizens has gone up, it is possible that the income of some citizens increased tremendously while the income of some others didn't increase, or at worst, actually decreased. In fact, it is normal that most of the country's GDP comes from a tiny percentage of its population. In the U.S., in 2013, 23 \% of the national income (closely related to the GDP) was due to incomes of just 1 \% of the population! Thus macroeconomic indicators may not indicate the well being of majority of the population! There are better indicators for measuring the well being of the majority of the population. One such is HDI, the \emph{Human Development Index}, a measure invented by Amartya Sen. Similar to how the purchasing power or inflation is measured, HDI uses a weighted sum of a basket of indicators such as life expectancy, literacy rate, crime rate, women's safety and factors indicative of the general standard of living. 
